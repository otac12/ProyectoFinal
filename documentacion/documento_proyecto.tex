\documentclass[12pt,a4paper]{article}
\usepackage[utf8]{inputenc}
\usepackage[T1]{fontenc}
\usepackage[spanish,es-tabla]{babel}
\usepackage{geometry}
\usepackage{amsmath}
\usepackage{graphicx}
\usepackage{xcolor}
\usepackage{listings}
\usepackage{booktabs}
\usepackage{longtable}
\usepackage{float}
\usepackage{caption}
\usepackage{subcaption}
\usepackage{tikz}
\usetikzlibrary{shapes,arrows,positioning,calc,arrows.meta}
\usepackage{pgfplots}
\pgfplotsset{compat=1.18}
\usepackage{etoolbox}
\usepackage[colorlinks=true,linkcolor=blue,urlcolor=blue,citecolor=blue]{hyperref}

\geometry{margin=2.5cm}

% Solución para conflictos babel-TikZ con caracteres especiales
\AtBeginEnvironment{tikzpicture}{\shorthandoff{<>}}
\AtEndEnvironment{tikzpicture}{\shorthandon{<>}}

% Configuración de código
\lstset{
    language=Python,
    basicstyle=\ttfamily\small,
    keywordstyle=\color{blue}\bfseries,
    commentstyle=\color{green!60!black},
    stringstyle=\color{red},
    numbers=left,
    numberstyle=\tiny\color{gray},
    stepnumber=1,
    numbersep=5pt,
    backgroundcolor=\color{gray!10},
    frame=single,
    breaklines=true,
    breakatwhitespace=true,
    tabsize=4,
    showstringspaces=false
}

\lstdefinestyle{javascript}{
    language=JavaScript,
    basicstyle=\ttfamily\small,
    keywordstyle=\color{blue}\bfseries,
    commentstyle=\color{green!60!black},
    stringstyle=\color{red},
    numbers=left,
    numberstyle=\tiny\color{gray}
}

% Colores personalizados
\definecolor{primaryblue}{RGB}{33,150,243}
\definecolor{primaryorange}{RGB}{255,152,0}
\definecolor{primarygreen}{RGB}{76,175,80}

% Metadata
\title{\textbf{Simulador de Red FTTH GPON}\\
       \large{Sistema Completo de Simulación y Análisis de Redes Ópticas}}
\author{Documentación Técnica del Proyecto}
\date{\today}

\begin{document}

\maketitle

\newpage
\tableofcontents
\newpage

% ============================================================================
% RESUMEN EJECUTIVO
% ============================================================================
\section{Resumen Ejecutivo}

Este documento presenta la documentación completa del \textbf{Simulador de Red FTTH GPON}, un sistema integral desarrollado para simular, planificar y analizar redes de fibra óptica FTTH (Fiber To The Home) utilizando tecnología GPON (Gigabit Passive Optical Network).

El sistema está compuesto por:
\begin{itemize}
    \item \textbf{Backend}: API REST desarrollada en Python/Flask con MySQL
    \item \textbf{Frontend}: Interfaz web interactiva en React.js
    \item \textbf{Simulador}: Motor de simulación de tráfico con algoritmo DBA (Dynamic Bandwidth Allocation)
    \item \textbf{Modelado}: Sistema completo de elementos de red óptica (OLT, ONU, Splitter, Fiber)
\end{itemize}

\subsection{Características Principales}
\begin{enumerate}
    \item Creación de topologías GPON personalizables (Star, Bus, Ring, Tree)
    \item Visualización interactiva de redes con diagramas
    \item Simulación de tráfico Triple Play (video, internet, voz)
    \item Cálculo de power budget óptico
    \item Algoritmo DBA IPACT para asignación de ancho de banda
    \item Métricas de rendimiento en tiempo real
    \item Base de datos persistente para configuraciones
\end{enumerate}

% ============================================================================
% INTRODUCCIÓN
% ============================================================================
\section{Introducción}

\subsection{Contexto Tecnológico}

GPON (Gigabit Passive Optical Network) es una tecnología de acceso de banda ancha que utiliza fibra óptica para proporcionar servicios de comunicación a usuarios finales. Esta tecnología permite:

\begin{itemize}
    \item Altas velocidades de transmisión (hasta 2.5 Gbps downstream)
    \item Distancias de hasta 20 km sin regeneración
    \item Soporte para múltiples servicios (Triple Play)
    \item Infraestructura pasiva (menor costo operativo)
\end{itemize}

\subsection{Objetivos del Proyecto}

El objetivo principal es desarrollar una herramienta de simulación que permita:

\begin{enumerate}
    \item \textbf{Diseño}: Crear y configurar topologías de red GPON
    \item \textbf{Análisis}: Calcular parámetros técnicos (power budget, pérdidas)
    \item \textbf{Simulación}: Modelar comportamiento de tráfico en la red
    \item \textbf{Optimización}: Evaluar diferentes configuraciones y escenarios
\end{enumerate}

% ============================================================================
% ARQUITECTURA DEL SISTEMA
% ============================================================================
\section{Arquitectura del Sistema}

\subsection{Arquitectura General}

El sistema sigue una arquitectura de tres capas:

\begin{figure}[H]
\centering
\begin{tikzpicture}[
    node distance=2cm,
    rect/.style={rectangle, draw=primaryblue, fill=primaryblue!20, thick, minimum width=3cm, minimum height=1.5cm, align=center, rounded corners},
    arrow/.style={->, >=stealth, thick}
]
    % Capa Presentación
    \node[rect, fill=primarygreen!30] (frontend) at (0,6) {Frontend\\React.js};
    
    % Capa Lógica
    \node[rect, fill=primaryblue!30] (backend) at (0,3.5) {Backend\\Flask API};
    
    % Capa Datos
    \node[rect, fill=primaryorange!30] (database) at (0,1) {Base de Datos\\MySQL};
    
    % Flechas (usar modo matemático para proteger caracteres especiales)
    \draw[arrow] (frontend) -- node[right] {\texttt{HTTP/REST}} (backend);
    \draw[arrow] (backend) -- node[right] {\texttt{SQL}} (database);
    \draw[arrow] (backend) to[bend left=30] node[above] {\texttt{JSON}} (frontend);
    \draw[arrow] (database) to[bend left=30] node[below] {Resultados} (backend);
    
\end{tikzpicture}
\caption{Arquitectura de tres capas del sistema}
\end{figure}

\subsection{Componentes Principales}

\subsubsection{Backend (Python/Flask)}

El backend está construido con:
\begin{itemize}
    \item \textbf{Flask}: Framework web ligero
    \item \textbf{SQLAlchemy}: ORM para gestión de base de datos
    \item \textbf{MySQL}: Sistema de gestión de base de datos
    \item \textbf{SimPy}: Librería de simulación discreta de eventos
    \item \textbf{NumPy}: Cálculos numéricos y estadísticos
\end{itemize}

\subsubsection{Frontend (React.js)}

El frontend utiliza:
\begin{itemize}
    \item \textbf{React 18}: Librería para interfaces de usuario
    \item \textbf{Material-UI}: Componentes de interfaz
    \item \textbf{ReactFlow}: Visualización de grafos y diagramas
    \item \textbf{Recharts}: Gráficos y visualizaciones
    \item \textbf{Axios}: Cliente HTTP
\end{itemize}

\subsubsection{Infraestructura}

El despliegue se realiza mediante:
\begin{itemize}
    \item \textbf{Docker}: Contenedores para cada servicio
    \item \textbf{Docker Compose}: Orquestación de servicios
    \item \textbf{MySQL 8.0}: Base de datos relacional
\end{itemize}

% ============================================================================
% MODELADO DE ELEMENTOS DE RED
% ============================================================================
\section{Modelado de Elementos de Red}

\subsection{Clases Principales}

El sistema modela los siguientes elementos de red GPON:

\subsubsection{OLT (Optical Line Terminal)}

El OLT es el terminal de línea óptica ubicado en la central del proveedor.

\begin{table}[H]
\centering
\caption{Características del OLT}
\begin{tabular}{@{}ll@{}}
\toprule
\textbf{Parámetro} & \textbf{Valor} \\
\midrule
Potencia de transmisión (TX) & 2.5 dBm \\
Sensibilidad de recepción (RX) & -27 dBm \\
Capacidad total & 2500 Mbps \\
Power Budget & 29.5 dB \\
\bottomrule
\end{tabular}
\end{table}

\subsubsection{ONU (Optical Network Unit)}

Las ONUs son los dispositivos finales ubicados en las instalaciones del cliente.

\begin{table}[H]
\centering
\caption{Características de la ONU}
\begin{tabular}{@{}ll@{}}
\toprule
\textbf{Parámetro} & \textbf{Valor} \\
\midrule
Potencia de transmisión (TX) & 1.5 dBm \\
Sensibilidad de recepción (RX) & -24 dBm \\
Rango de tráfico & 0.064 - 25 Mbps \\
\bottomrule
\end{tabular}
\end{table}

\subsubsection{Splitter Óptico}

El splitter divide la señal óptica entre múltiples ONUs.

\begin{table}[H]
\centering
\caption{Pérdidas por Splitting}
\begin{tabular}{@{}cc@{}}
\toprule
\textbf{Ratio} & \textbf{Pérdida (dB)} \\
\midrule
1:8 & 9.0 \\
1:16 & 12.0 \\
1:32 & 15.0 \\
1:64 & 18.0 \\
\bottomrule
\end{tabular}
\end{table}

La fórmula utilizada es:
\begin{equation}
L_{split} = 10 \times \log_{10}(N)
\end{equation}

donde \(N\) es el número de puertos de salida.

\subsubsection{Fibra Óptica}

Las fibras tienen las siguientes características:

\begin{table}[H]
\centering
\caption{Parámetros de Fibra Óptica}
\begin{tabular}{@{}ll@{}}
\toprule
\textbf{Parámetro} & \textbf{Valor} \\
\midrule
Atenuación típica & 0.2 dB/km \\
Pérdida por empalme & 0.1 dB cada 2 km \\
Dispersión & 17 ps/(nm·km) \\
\bottomrule
\end{tabular}
\end{table}

% ============================================================================
% CÓDIGO EXPLICADO
% ============================================================================
\section{Explicación del Código}

\subsection{Creación de Topología en Estrella}

El código siguiente muestra cómo se crea una topología GPON estándar en estrella:

\begin{lstlisting}[caption=Creación de topología en estrella]
def _create_star_topology(self, num_onus=32, split_ratio="1:32"):
    """Generar topología en estrella (GPON estándar)"""
    # Crear OLT
    self.olt = OLT(id="OLT-1", name="OLT Principal")
    
    # Crear splitter central
    splitter = Splitter(
        id="SPLIT-1", 
        name=f"Splitter {split_ratio}", 
        ratio=split_ratio
    )
    self.add_splitter(splitter)
    self.olt.connect_splitter(splitter)
    
    # Crear fibra OLT-Splitter
    fiber_olt_splitter = OpticalFiber(
        id="FIBER-OLT-SPLIT",
        name="OLT to Splitter",
        length=2.0  # 2 km
    )
    fiber_olt_splitter.connect(self.olt, splitter)
    self.add_fiber(fiber_olt_splitter)
    
    # Crear ONUs conectadas al splitter central
    for i in range(num_onus):
        onu = ONU(id=f"ONU-{i+1}", name=f"ONU {i+1}")
        self.add_onu(onu)
        splitter.connect_onu(onu)
        
        # Crear fibra Splitter-ONU
        fiber = OpticalFiber(
            id=f"FIBER-{i+1}",
            name=f"Splitter to ONU {i+1}",
            length=3.0 + (i * 0.1)  # Variar longitudes
        )
        fiber.connect(splitter, onu)
        self.add_fiber(fiber)
\end{lstlisting}

\subsection{Cálculo de Power Budget}

El cálculo del power budget es crítico para validar que la señal óptica llegue con suficiente potencia:

\begin{lstlisting}[caption=Cálculo de power budget]
def calculate_power_budget_path(self, onu):
    """Calcular power budget para una ruta OLT-ONU"""
    if not self.olt or not onu.connected_splitter:
        return None
    
    # Calcular power budget base
    power_budget = self.olt.calculate_power_budget()
    # power_budget = TX_power - RX_sensitivity
    
    # Pérdida en splitter
    splitter_loss = onu.connected_splitter.split_loss
    
    # Pérdidas en fibras
    total_fiber_loss = 0
    total_splice_loss = 0
    
    for fiber in self.fibers:
        if fiber.to_element == onu:
            fiber_attenuation = fiber.calculate_loss()
            splice_loss = 0.1 * (int(fiber.length / 2) + 1)
            total_fiber_loss += fiber_attenuation
            total_splice_loss += splice_loss
    
    total_loss = splitter_loss + total_fiber_loss + total_splice_loss
    
    # Margen de seguridad (3 dB)
    margin = 3.0
    available_power = power_budget - total_loss - margin
    
    return {
        "power_budget": power_budget,
        "total_loss": total_loss,
        "margin": margin,
        "available_power": available_power,
        "is_valid": available_power >= 0
    }
\end{lstlisting}

La fórmula completa es:
\begin{equation}
P_{available} = (P_{TX} - P_{RX}) - L_{split} - L_{fiber} - L_{splice} - M
\end{equation}

donde:
\begin{itemize}
    \item \(P_{TX}\): Potencia de transmisión del OLT
    \item \(P_{RX}\): Sensibilidad de recepción del OLT
    \item \(L_{split}\): Pérdida por splitting
    \item \(L_{fiber}\): Pérdida en fibras
    \item \(L_{splice}\): Pérdida por empalmes
    \item \(M\): Margen de seguridad (3 dB)
\end{itemize}

\subsection{Algoritmo DBA (Dynamic Bandwidth Allocation)}

El algoritmo DBA implementa la estrategia IPACT (Interleaved Polling with Adaptive Cycle Time):

\begin{lstlisting}[caption=Algoritmo DBA IPACT]
def allocate_bandwidth(self, onu_requests, total_capacity=None):
    """Asignar ancho de banda usando estrategia IPACT"""
    if total_capacity is None:
        total_capacity = self.total_capacity
    
    allocation = {}
    remaining = total_capacity
    
    # IPACT: asignar en orden de llegada
    sorted_requests = sorted(onu_requests.items())
    
    for onu_id, requested in sorted_requests:
        # Asignar mínimo entre lo solicitado y lo disponible
        granted = min(requested, remaining)
        allocation[onu_id] = {
            'requested': requested,
            'granted': granted,
            'utilization': (granted / requested * 100) 
                          if requested > 0 else 0
        }
        remaining -= granted
        
        if remaining <= 0:
            # Si no hay más capacidad, asignar 0 a las restantes
            break
    
    return {
        'allocations': allocation,
        'total_granted': sum(a['granted'] 
                            for a in allocation.values()),
        'global_utilization': (total_granted / total_capacity * 100)
    }
\end{lstlisting}

\subsection{Simulador de Tráfico}

El simulador genera tráfico Triple Play con diferentes patrones:

\begin{lstlisting}[caption=Generación de tráfico Triple Play]
def setup_onus(self, traffic_profiles=None):
    """Configurar generadores de tráfico para ONUs"""
    if traffic_profiles is None:
        # Perfiles por defecto: Triple Play
        for i in range(self.num_onus):
            # Mezcla de servicios
            service_type = random.choice([
                'video', 'internet', 'voice'
            ])
            
            if service_type == 'video':
                rate = random.uniform(10, 25)  # Mbps
                pattern = 'constant'
            elif service_type == 'internet':
                rate = random.uniform(5, 15)   # Mbps
                pattern = 'poisson'
            else:  # voice
                rate = random.uniform(0.064, 0.1)  # Mbps
                pattern = 'constant'
            
            generator = TrafficGenerator(
                self.env,
                f'ONU-{i+1}',
                pattern,
                rate
            )
            self.generators.append(generator)
            self.env.process(generator.traffic_process())
\end{lstlisting}

% ============================================================================
% DIAGRAMAS DE TOPOLOGÍAS
% ============================================================================
\section{Diagramas de Topologías}

\subsection{Topología en Estrella (Star)}

La topología en estrella es la más común en GPON:

\begin{figure}[H]
\centering
\begin{tikzpicture}[
    node distance=1.5cm,
    olt/.style={circle, draw=primaryblue, fill=primaryblue!30, 
                thick, minimum size=1cm},
    splitter/.style={rectangle, draw=primaryorange, fill=primaryorange!30, 
                     thick, minimum size=0.8cm, rounded corners},
    onu/.style={circle, draw=primarygreen, fill=primarygreen!30, 
                thick, minimum size=0.6cm},
    fiber/.style={draw=gray, thick}
]
    % OLT
    \node[olt] (olt) at (0, 4) {OLT};
    
    % Splitter
    \node[splitter] (split) at (0, 2) {SPL};
    
    % ONUs (8 para claridad)
    \foreach \i in {1,...,8} {
        \pgfmathsetmacro{\angle}{(\i-1)*360/8}
        \pgfmathsetmacro{\x}{2*cos(\angle)}
        \pgfmathsetmacro{\y}{0.5*sin(\angle)}
        \node[onu] (onu\i) at (\x, \y) {ONU\i};
        \draw[fiber] (split) -- (onu\i);
    }
    
    % Conexión OLT-Splitter
    \draw[fiber] (olt) -- (split);
    
    % Etiquetas
    \node at (0, 3.2) {\tiny Fibra OLT-Splitter};
    \node at (0, 1.2) {\tiny Splitter 1:32};
    
\end{tikzpicture}
\caption{Topología GPON en estrella con splitter central}
\end{figure}

\subsection{Topología en Árbol (Tree)}

La topología en árbol permite múltiples niveles de splitters:

\begin{figure}[H]
\centering
\begin{tikzpicture}[
    node distance=1.2cm,
    olt/.style={circle, draw=primaryblue, fill=primaryblue!30, 
                thick, minimum size=1cm},
    splitter/.style={rectangle, draw=primaryorange, fill=primaryorange!30, 
                     thick, minimum size=0.7cm, rounded corners},
    onu/.style={circle, draw=primarygreen, fill=primarygreen!30, 
                thick, minimum size=0.5cm},
    fiber/.style={draw=gray, thick}
]
    % OLT
    \node[olt] (olt) at (0, 5) {OLT};
    
    % Splitter raíz
    \node[splitter] (root) at (0, 4) {Root};
    
    % Splitters intermedios
    \node[splitter] (split1) at (-2, 3) {S1};
    \node[splitter] (split2) at (0, 3) {S2};
    \node[splitter] (split3) at (2, 3) {S3};
    
    % ONUs - Primera fila
    \node[onu] (onu1) at (-2.5, 1.5) {ONU1};
    \node[onu] (onu2) at (0, 1.5) {ONU2};
    \node[onu] (onu3) at (2.5, 1.5) {ONU3};
    
    % ONUs - Segunda fila
    \node[onu] (onu4) at (-2.5, 1) {ONU4};
    \node[onu] (onu5) at (0, 1) {ONU5};
    \node[onu] (onu6) at (2.5, 1) {ONU6};
    
    % Conexiones desde splitters
    \draw[fiber] (split1) -- (onu1);
    \draw[fiber] (split1) -- (onu4);
    \draw[fiber] (split2) -- (onu2);
    \draw[fiber] (split2) -- (onu5);
    \draw[fiber] (split3) -- (onu3);
    \draw[fiber] (split3) -- (onu6);
    
    % Conexiones
    \draw[fiber] (olt) -- (root);
    \draw[fiber] (root) -- (split1);
    \draw[fiber] (root) -- (split2);
    \draw[fiber] (root) -- (split3);
    
\end{tikzpicture}
\caption{Topología GPON en árbol con múltiples niveles}
\end{figure}

\subsection{Flujo de Datos en GPON}

\begin{figure}[H]
\centering
\begin{tikzpicture}[
    node distance=1.5cm,
    box/.style={rectangle, draw=black, fill=gray!10, 
                thick, minimum width=3cm, minimum height=1cm, 
                align=center},
    arrow/.style={->, >=stealth, thick, blue}
]
    \node[box] (onu1) {ONU 1};
    \node[box] (onu2) [below of=onu1] {ONU 2};
    \node[box] (onu3) [below of=onu2] {ONU N};
    
    \node[box, fill=orange!30] (splitter) [right of=onu2, xshift=2cm] {Splitter};
    
    \node[box, fill=blue!30] (olt) [right of=splitter, xshift=2cm] {OLT};
    
    \draw[arrow] (onu1) -- node[above] {Upstream} (splitter);
    \draw[arrow] (onu2) -- (splitter);
    \draw[arrow] (onu3) -- node[below] {TDM} (splitter);
    
    \draw[arrow] (splitter) -- (olt);
    
    \draw[arrow, dashed] (olt) -- node[above] {Downstream} (splitter);
    \draw[arrow, dashed] (splitter) -- (onu1);
    \draw[arrow, dashed] (splitter) -- (onu2);
    \draw[arrow, dashed] (splitter) -- (onu3);
    
    \node at (-1, -1.5) {Upstream: TDM};
    \node at (4, -1.5) {Downstream: Broadcast};
    
\end{tikzpicture}
\caption{Flujo bidireccional de datos en GPON}
\end{figure}

% ============================================================================
% ARQUITECTURA DE LA APLICACIÓN
% ============================================================================
\section{Arquitectura de la Aplicación}

\subsection{Diagrama de Clases}

\begin{figure}[H]
\centering
\begin{tikzpicture}[
    class/.style={rectangle, draw=black, fill=blue!10, 
                  thick, minimum width=4cm, minimum height=1.5cm, 
                  align=center, rounded corners},
    arrow/.style={->, >=stealth, thick}
]
    \node[class] (network) {OpticalNetwork};
    \node[class] (olt) [below left of=network, xshift=-1cm, yshift=-1cm] {OLT};
    \node[class] (onu) [below of=network, yshift=-1cm] {ONU};
    \node[class] (splitter) [below right of=network, xshift=1cm, yshift=-1cm] {Splitter};
    \node[class] (fiber) [right of=network, xshift=2cm] {OpticalFiber};
    
    \draw[arrow] (network) -- node[left] {contiene} (olt);
    \draw[arrow] (network) -- node[left] {contiene} (onu);
    \draw[arrow] (network) -- node[left] {contiene} (splitter);
    \draw[arrow] (network) -- node[above] {contiene} (fiber);
    \draw[arrow] (olt) -- node[above, sloped] {conecta} (splitter);
    \draw[arrow] (splitter) -- node[above, sloped] {conecta} (onu);
    \draw[arrow] (fiber) -- node[below, sloped] {une} (olt);
    \draw[arrow] (fiber) -- node[below, sloped] {une} (splitter);
    \draw[arrow] (fiber) -- node[below, sloped] {une} (onu);
    
\end{tikzpicture}
\caption{Diagrama de clases simplificado del modelo de red}
\end{figure}

\subsection{Flujo de Datos en la API}

\begin{figure}[H]
\centering
\begin{tikzpicture}[
    process/.style={rectangle, draw=black, fill=green!20, 
                    thick, minimum width=3cm, minimum height=1cm, 
                    align=center, rounded corners},
    data/.style={rectangle, draw=black, fill=yellow!20, 
                 thick, minimum width=2.5cm, minimum height=1cm, 
                 align=center, rounded corners},
    arrow/.style={->, >=stealth, thick}
]
    \node[process] (client) {Cliente React};
    \node[process] (api) [right of=client, xshift=2cm] {Flask API};
    \node[process] (db) [right of=api, xshift=2cm] {MySQL};
    \node[process] (sim) [below of=api, yshift=-1.5cm] {Simulador};
    
    \node[data] (json1) [below of=client, yshift=-0.5cm] {JSON Request};
    \node[data] (json2) [above of=client, yshift=0.5cm] {JSON Response};
    
    \draw[arrow] (client) -- (api);
    \draw[arrow] (api) -- (db);
    \draw[arrow] (db) -- (api);
    \draw[arrow] (api) -- (sim);
    \draw[arrow] (sim) -- (api);
    \draw[arrow] (api) -- (client);
    
\end{tikzpicture}
\caption{Flujo de datos en la aplicación}
\end{figure}

% ============================================================================
% TABLAS DE ESPECIFICACIONES
% ============================================================================
\section{Especificaciones Técnicas}

\subsection{Especificaciones del Sistema}

\begin{table}[H]
\centering
\caption{Requisitos del Sistema}
\begin{tabular}{@{}ll@{}}
\toprule
\textbf{Componente} & \textbf{Especificación} \\
\midrule
Python & $\geq$ 3.8 \\
Node.js & $\geq$ 16.0 \\
MySQL & 8.0 \\
Docker & $\geq$ 20.0 \\
RAM Mínima & 4 GB \\
Puerto Frontend & 3001 \\
Puerto Backend & 5000 \\
Puerto MySQL & 3306 \\
\bottomrule
\end{tabular}
\end{table}

\subsection{Endpoints de la API}

\begin{longtable}{@{}p{4cm}p{7cm}p{3cm}@{}}
\toprule
\textbf{Endpoint} & \textbf{Descripción} & \textbf{Método} \\
\midrule
\endfirsthead
\toprule
\textbf{Endpoint} & \textbf{Descripción} & \textbf{Método} \\
\midrule
\endhead
\midrule
\multicolumn{3}{r}{{Continúa en la siguiente página}} \\
\endfoot
\bottomrule
\endlastfoot
/api/health & Verificación de salud del servidor & GET \\
/api/topologies & Obtener todas las topologías & GET \\
/api/topologies & Crear nueva topología & POST \\
/api/topologies/<id> & Obtener topología específica & GET \\
/api/topologies/<id>/power-budget & Calcular power budget & GET \\
/api/simulations & Crear y ejecutar simulación & POST \\
/api/simulations/<id> & Obtener resultados de simulación & GET \\
/api/dba/allocate & Calcular asignación DBA & POST \\
\end{longtable}

\subsection{Modelo de Base de Datos}

\begin{table}[H]
\centering
\caption{Estructura de Tablas}
\begin{tabular}{@{}lp{8cm}@{}}
\toprule
\textbf{Tabla} & \textbf{Descripción} \\
\midrule
network\_topologies & Almacena configuraciones de topologías \\
network\_elements & Almacena elementos individuales (OLT, ONU, etc.) \\
simulations & Almacena ejecuciones de simulaciones \\
performance\_metrics & Almacena métricas de rendimiento por ONU \\
\bottomrule
\end{tabular}
\end{table}

% ============================================================================
% CÁLCULOS Y FÓRMULAS
% ============================================================================
\section{Fórmulas y Cálculos}

\subsection{Power Budget}

El power budget total se calcula como:

\begin{equation}
PB_{total} = P_{TX,OLT} - P_{RX,OLT}
\end{equation}

\subsection{Pérdidas en la Red}

Las pérdidas totales incluyen:

\begin{equation}
L_{total} = L_{splitter} + L_{fiber} + L_{splice} + M
\end{equation}

donde:
\begin{align}
L_{fiber} &= \alpha \times L \\
L_{splice} &= 0.1 \times \left\lfloor \frac{L}{2} \right\rfloor \\
L_{splitter} &= 10 \times \log_{10}(N)
\end{align}

\subsection{Disponibilidad de Potencia}

La potencia disponible al final del enlace es:

\begin{equation}
P_{available} = PB_{total} - L_{total}
\end{equation}

La red es válida si \(P_{available} \geq 0\).

\subsection{Throughput}

El throughput de una ONU se calcula como:

\begin{equation}
Throughput = \frac{Bytes \times 8}{Tiempo} \quad \text{(Mbps)}
\end{equation}

\subsection{Utilización del Ancho de Banda}

La utilización global del sistema:

\begin{equation}
U_{global} = \frac{BW_{asignado}}{BW_{total}} \times 100\%
\end{equation}

% ============================================================================
% CONCLUSIONES
% ============================================================================
\section{Conclusiones}

Este proyecto implementa un simulador completo de redes FTTH GPON que permite:

\begin{enumerate}
    \item \textbf{Diseño eficiente}: Crear topologías personalizadas con validación automática
    \item \textbf{Análisis preciso}: Cálculo de power budget y pérdidas ópticas
    \item \textbf{Simulación realista}: Modelado de tráfico Triple Play con diferentes patrones
    \item \textbf{Optimización}: Algoritmo DBA para asignación dinámica de ancho de banda
    \item \textbf{Visualización}: Interfaz interactiva para análisis y presentación
\end{enumerate}

\subsection{Mejoras Futuras}

Posibles extensiones del sistema:

\begin{itemize}
    \item Implementación de algoritmos DBA avanzados (EPON, XG-PON)
    \item Modelado de pérdidas por conexiones y conectores
    \item Simulación de fallos y redundancia
    \item Optimización automática de topologías
    \item Integración con herramientas de planificación GIS
\end{itemize}

% ============================================================================
% REFERENCIAS
% ============================================================================
\section{Referencias}

\begin{itemize}
    \item ITU-T G.984.1-3: Gigabit-capable Passive Optical Networks (GPON)
    \item ITU-T G.984.4: OMCI (ONU Management and Control Interface)
    \item IEEE 802.3ah: Ethernet Passive Optical Network (EPON)
    \item Flask Documentation: \url{https://flask.palletsprojects.com/}
    \item React Documentation: \url{https://react.dev/}
    \item SimPy Documentation: \url{https://simpy.readthedocs.io/}
\end{itemize}

% ============================================================================
% APÉNDICES
% ============================================================================
\appendix

\section{Instalación y Configuración}

\subsection{Requisitos Previos}

\begin{lstlisting}[language=bash, caption=Comandos de instalación]
# Clonar repositorio
git clone <url-repositorio>
cd Proyecto

# Iniciar servicios con Docker Compose
docker-compose up --build
\end{lstlisting}

\subsection{Configuración de Variables de Entorno}

\begin{lstlisting}[caption=Variables de entorno]
MYSQL_HOST=mysql
MYSQL_PORT=3306
MYSQL_USER=ftth_user
MYSQL_PASSWORD=ftth_password
MYSQL_DATABASE=ftth_db
FLASK_APP=app.py
REACT_APP_API_URL=http://localhost:5000
\end{lstlisting}

\section{Glosario de Términos}

\begin{description}
    \item[FTTH] Fiber To The Home - Fibra hasta el hogar
    \item[GPON] Gigabit Passive Optical Network - Red óptica pasiva de gigabits
    \item[OLT] Optical Line Terminal - Terminal de línea óptica
    \item[ONU] Optical Network Unit - Unidad de red óptica
    \item[DBA] Dynamic Bandwidth Allocation - Asignación dinámica de ancho de banda
    \item[IPACT] Interleaved Polling with Adaptive Cycle Time
    \item[Power Budget] Presupuesto de potencia óptica disponible
    \item[Splitter] Divisor óptico pasivo
    \item[TDM] Time Division Multiplexing - Multiplexación por división de tiempo
\end{description}

\end{document}

